% \documentclass[1pt]{article}
\documentclass[preview]{standalone}
% \documentclass[border=2pt]{standalone}
\usepackage[a5paper]{geometry}

\title{algorithm}

% From the package documentation:
% "The package algorithmicx itself doesn’t define any
% algorithmic commands, but gives a set of macros to 
% define such a command set.  You may use only 
% algorithmicx, and define the commands yourself, or you 
% may use one of the predefined command sets."
% 
% Popular predefined command sets include algpseudocode and algpascal. algcompatible should only be used in old documents, and algc is incomplete.

% \usepackage{algpseudocode}
\usepackage[noend]{algorithmicx}
\usepackage[noline,boxruled,commentsnumbered,linesnumbered,titlenumbered]{algpseudocode}

\algrenewcommand\algorithmicindent{1.0em}%
% \usepackage{algorithm2e}
\algtext*{EndFor}% Remove "end while" text
\algtext*{EndIf}% Remove "end if" text
\algtext*{EndFunction}% Remove "end if" 
\algtext*{EndProcedure}% Remove "end procedure" text
% \usepackage[options ]{algorithm2e}


% \usepackage{xpatch}

% \makeatletter
% \xpatchcmd{\algorithmic}
%   {\ALG@tlm\z@}{\leftmargin\z@\ALG@tlm\z@}
%   {}{}
% \makeatother

% \newlength\myindent
% \setlength\myindent{2em}
% \newcommand\bindent{%
%   \begingroup
%   \setlength{\itemindent}{\myindent}
%   \addtolength{\algorithmicindent}{\myindent}
% }
% \newcommand\eindent{\endgroup}

\newcommand{\pluseq}{\mathrel{+}=}
\let\oldReturn\Return
\renewcommand{\Return}{\State\oldReturn}

\algnewcommand\algorithmicforeach{\textbf{for each}}
\algdef{S}[FOR]{ForEach}[1]{\algorithmicforeach\ #1\ \algorithmicdo}


\begin{document}

% \noindent
% With \verb|\SetInd{0.25em}{0.1em}|:
% \SetInd{0.25em}{0.1em}

% LinearPartition Algorithm: Simplified Version
\begin{algorithmic}[1]
% \begin{algorithm}[H]
% \SetAlgoLined
% \Require $n \geq 0$
% \Ensure $y = x^n$
\newcommand{\INDSTATE}[1][1]{\STATE\hspace{#1\algorithmicindent}}

\Procedure{LinearPartition}{$\mathbf x$}
% \bindent
    \State $n \gets$ length of $\mathbf x$
    \State $Q \gets$ hash() \Comment{hash table: from key $(i,j)$ to score}
    \State $Q(0,1) \gets 1$ \Comment{axiom} 
    \For{$j=1,...n$}
        % \State $C(0,j+1) \gets C(0,j) \cdot e^{-\frac{\delta(\mathbf{x},j)}{RT}} $ %\Comment{Initialize}
        \State $Q(j,j+1) \gets 1$ \Comment{PUSH} % 1 = e{-(0/RT)}
        % \State $C(0,j+1) \gets C(0,j)$ \Comment{Initialize}
        \ForEach{key $(i,j)$ in $Q$}
            \State $Q(i,j+1) \gets Q(i,j) \cdot e^{-\frac{\delta(\mathbf{x},j)}{RT}} $ \Comment{SKIP}
            \If{$(x_i,x_j)$ in \{AU, UA, CG, GC, GU, UG\}} 
                % \State $Q_{i,j+1} \gets  C(i,j) \cdot e^{-\frac{\xi(\mathbf{x},i,j)}{RT}} $
                \ForEach{key $(k,i)$ in $Q$}
                    \State $Q(k,j+1) \pluseq {Q(k,i) \cdot Q(i,j) \cdot e^{-\frac{\xi(\mathbf{x},i,j)}{RT}}} $ \Comment{POP}
                    % \State $C(0,j+1) \pluseq {C(0,k) \cdot C(k,j+1) \cdot e^{-\frac{\xi(\mathbf{x},i,j)}{RT}}} $
                \EndFor
                % \State $C(0,j+1) \pluseq C(0,i) \cdot Q_{i,j+1}$ \Comment{COMBINE}
            \EndIf
        \EndFor
        \State $\textsc {BeamPrune}(Q,j+1, \textrm {beamsize})$
    \EndFor
    \Return $Q(0,n+1)$
% \eindent
\EndProcedure

% \\

% \Function{beamprune}{$Q, j, b$}
%     \State $cands \gets$ hash() \Comment{hash table: from candidates $i$ to score}
%     \ForEach{key $(i,j)$ in $Q$}
%         \State $cands(i) \gets Q(0,i) \cdot Q(i,j)$ \Comment{$Q(0,i) $ as prefix score}
%     \EndFor
%     \State $cands \gets$ SelectTopB($cands, b$) \Comment{select top-$b$ by score}
%     \ForEach{key $(i,j)$ in $Q$}
%         \If{key $i$ not in $cands$}
%             \State delete $(i,j)$ in $Q$ \Comment{prune out low-scoring states}
%         \EndIf
%     \EndFor
% \EndFunction
        




% \State $y \gets 1$
% \State $X \gets x$
% \State $N \gets n$
% \While{$N \neq 0$}
% \If{$N$ is even}
%   \State $X \gets X \times X$
%   \State $N \gets \frac{N}{2} $  \Comment{This is a comment}
% \ElsIf{$N$ is odd}
%   \State $y \gets y \times X$
%   \State $N \gets N - 1$
% \EndIf
% \EndWhile
\end{algorithmic}
% \end{algorithm}

\end{document}
