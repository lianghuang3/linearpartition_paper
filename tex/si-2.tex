\onecolumn
\newpage
%\begin{titlepage}
  \begin{centering}
    \vspace*{1cm}
    
    \textbf{\Huge Supporting Information}\\
    \vspace{0.5cm}
    \textbf{\Huge LinearFold: Linear-Time Prediction of RNA Secondary Structure}\\
    \vspace{0.5cm}
    \textbf{\large Dezhong Deng, Kai Zhao, David Hendrix, David H. Mathews, and
      Liang Huang}
    \vspace{1.5cm}
    
  \end{centering}
%\end{titlepage}

\setcounter{figure}{0}
\renewcommand{\thefigure}{SI\,\arabic{figure}} % spacing
\setcounter{table}{0}
\renewcommand{\thetable}{SI\,\arabic{table}}


\begin{figure}[h]
  \centering
  \scalebox{1.2}{
  \begin{tabular}{lll}
    \toprule
    Code & \myurl{https://github.com/abentu0101/LinearFold} & in C++; only requires GCC 4.9.0+\\
    \midrule
    Server  & \myurl{http://linearfold.eecs.oregonstate.edu} & world's fastest; sequence limit 250,000\\
    \midrule
    Demo  & \myurl{http://linearfold.eecs.oregonstate.edu/preset.html} & prediction results for various beam sizes\\
    \bottomrule
  \end{tabular}
  }
  \caption {LinearFold source code and web server availability. We have the world's fastest RNA secondary structure prediction server, and the longest sequence limit (by comparison,
  Vienna RNAfold and RNAstructure have sequence limits of 10,000 and 4,000, respectively.
  Our \linearfold server has been used by researchers from 27 countries.}
  \label{fig:codeserver}
\end{figure}

%% \iffalse
%\begin{table*}[h]
%\centering
%\begin{tabular}{r|rrr||rr|rr||rr|rr}
%       & \multicolumn{2}{c}{\# of sequences} & avg. & \multicolumn{2}{c|}{\contrafoldmfe} & \multicolumn{2}{c||}{\linearfoldc $b$=100}
%                                                                                                                              & \multicolumn{2}{c|}{\viennarnafold} & \multicolumn{2}{c}{\linearfoldv $b$=100}\\
%Family & total & used & length & 
%                                        PPV & sensitivity & $\Delta$PPV & $\Delta$sensitivity & 
%                                                                                                PPV & sensitivity & $\Delta$PPV & $\Delta$sensitivity \\
%\hline
%tRNA & 557 & 74 & 77.3 & 69.32 & 70.84 & +0.00 & +0.00 & 63.80 & 73.25 & -0.13 & +0.00 \\
%5S rRNA & 1,283 & 1,125 & 118.8 & 76.48 & 76.62 & +0.13 & +0.01 & 60.12 & 66.62 & +0.01 & +0.04 \\
%SRP  & 928 & 886 & 186.1 & 64.24 & 63.88 & -0.03 & -0.03 & 60.93 & 66.66 & (*)+0.38 & (*)+0.29 \\
%RNaseP & 454 & 182 & 344.1 & 49.34 & 48.10 & -0.15 & -0.17 & 47.78 & 55.79 & +0.03 & -0.19 \\
%tmRNA  & 462 & 462 & 366.0 & 46.50 & 39.70 & (*)-0.62 & (*)-0.76 & 42.09 & 47.58 & (*)-0.96 & (*)-1.00 \\
%Group I Intron & 98 & 96 & 424.9 & 52.50 & 50.78 & (**)+1.30 & (*)+0.94 & 47.02 & 57.96 & (*)+0.80 & (*)+0.94 \\
%telomerase RNA & 37 & 37 & 444.6 & 46.12 & 59.96 & -0.07 & -0.05 & 41.87 & 58.77 & -0.01 & -0.14 \\
%%Group II Intron & 11 & 11 & 716.5 & 27.84 & 30.41 & -0.24 & +1.27 & 24.47 & 41.12 & +0.95 & +2.23 \\
%16S rRNA & 22 & 22 & 1,547.9 & 40.46 & 41.48 & (*)+3.89 & (*)+3.08 & 37.63 & 44.60 & (*)+1.46 & (*)+1.56 \\
%23S rRNA & 5 & 5 & 2,927.4 & 48.29 & 49.85 & (**)+14.00 & (**)+9.98 & 55.09 & 62.67 & +0.40 & +0.25 \\
%\hline
%{\em Overall} & 3,846 & 2,889 & 222.2 & 54.80 & 55.69 & +2.06 & +1.44 & 50.70 & 59.32 & +0.22 & +0.20 \\
%\end{tabular}
%\smallskip
%\caption{Detailed prediction accuracies in percent, allowing one nucleotide in a pair to be displaced by one position, on the ArchiveII dataset using \contrafoldmfe, \linearfoldc, 
%                  \viennarnafold and \linearfoldv.
%                              This slipping method~\cite{sloma+mathews:2016}  considers a base pair to
%                  be correct if it is slipped by one nucleotide on a strand. 
%         Statistical significance are marked by 
%         *($0.01\leq p<0.05$) and **($p<0.01$).
%         Overall, \linearfoldc outperforms \contrafoldmfe by +2.06 PPV and +1.44 sensitivity,
%                  and \linearfoldv outperforms \viennarnafold by +0.22 PPV and +0.20 sensitivity.
%         Among the nine families, \linearfoldc is significantly better on three (Group I Intron, 16S and 23S rRNAs),
%               and \linearfoldv is significantly better on three as well (SRP, Group I Intron, and 16S rRNAs).
%         We also report the accuracies using exact base pair match in the next Table.
%         \label{tab:accuracy}}
%\end{table*}

% \iffalse
\begin{table*}[h] % hzhang: redo with new data
  \centering
  \begin{tabular}{r|rrr||rr|rr||rr|rr}
    & \multicolumn{2}{c}{\# of sequences} & avg. & \multicolumn{2}{c|}{\contrafoldmfe} & \multicolumn{2}{c||}{\linearfoldc $b$=100}
    & \multicolumn{2}{c|}{\viennarnafold} & \multicolumn{2}{c}{\linearfoldv $b$=100}\\
    Family & total & used & length & 
    PPV & sensitivity & $\Delta$PPV & $\Delta$sensitivity & 
    PPV & sensitivity & $\Delta$PPV & $\Delta$sensitivity \\
    \hline
    tRNA & 557 & 74 & 77.3 & 68.89 & 70.54 & +0.00 & +0.00 & 63.51 & 72.92 & +0.24 & +0.19  \\
    5S rRNA & 1,283 & 1,125 & 118.8 & 73.66 & 73.74 & +0.00 & +0.00 & 59.55 & 65.96 & +0.03 & +0.04  \\
    SRP  & 928 & 886 & 186.1 & 62.73 & 62.41 & -0.07 & -0.07 & 59.91 & 65.42 & (*)+0.35 & +0.27 \\
    RNaseP & 454 & 182 & 344.1 & 48.91 & 47.90 &-0.22 &(*)-0.54 &47.28 & 55.15 &+0.12 &-0.07  \\
    tmRNA  & 462 & 462 & 366.0 &  44.88 & 38.61 &(*)-0.74 &(**)-0.93 &41.47 & 46.86 &(**)-0.95 &(**)-1.02  \\
    Group I Intron & 98 & 96 & 424.9 &  52.62 & 50.93 &+0.84 &(*)+0.80 &46.81 & 57.68 &(**)+0.86 &(*)+1.02  \\
    telomerase RNA & 37 & 37 & 444.6 &   45.39 & 59.19 &-0.05 &-0.11 &41.47 & 58.20 &+0.05 &-0.05\\
    16S rRNA & 22 & 22 & 1,547.9 &  41.08 & 41.77 &(*)+3.56 &(*)+3.09 &37.23 & 44.13 &(*)+1.51 &+1.59 \\
    23S rRNA & 5 & 5 & 2,927.4 &  52.47 & 53.18 &(*)+8.65 &(*)+5.66 &54.79 & 62.32 &+0.33 &+0.16 \\
    \hline
    {\em Overall} & 3,846 & 2,889 & 222.2  & 54.51 & 55.36 &+1.33  &+0.88  & 50.22 & 58.74 & +0.28&+0.24 \\
  \end{tabular}
  \smallskip
  \caption{Detailed prediction accuracies in percent, allowing one nucleotide in a pair to be displaced by one position, on the ArchiveII dataset using \contrafoldmfe, \linearfoldc, 
    \viennarnafold and \linearfoldv.
    This slipping method~\cite{sloma+mathews:2016}  considers a base pair to
    be correct if it is slipped by one nucleotide on a strand. 
    Statistical significance are marked by 
    *($0.01\leq p<0.05$) and **($p<0.01$).
    Overall, \linearfoldc outperforms \contrafoldmfe by +1.82 PPV and +1.23 sensitivity, % hzhang: redo with new data
    and \linearfoldv outperforms \viennarnafold by +0.28 PPV and +0.24 sensitivity.
    Among the nine families, \linearfoldc is significantly better on five (SRP, Group I Intron, 5S, 16S and 23S rRNAs), % hzhang: grp1 and 16s are significant only on PPV
    and \linearfoldv is significantly better on three (SRP, Group I Intron, and 16S rRNAs). % hzhang: srp is significant only on PPV
    We also report the accuracies using exact base pair match in the next Table.
    \label{tab:accuracy}}
\end{table*}

%\begin{table*}[h]
%  \centering
%  \begin{tabular}{r|rrr||rr|rr||rr|rr}
%    & \multicolumn{2}{c}{\# of sequences} & avg. & \multicolumn{2}{c|}{\contrafoldmfe} & \multicolumn{2}{c||}{\linearfoldc $b$=100} & \multicolumn{2}{c|}{\viennarnafold} & \multicolumn{2}{c}{\linearfoldv $b$=100}\\
%    Family & total & used & length & PPV & sensitivity & $\Delta$PPV & $\Delta$sensitivity & PPV & sensitivity & $\Delta$PPV & $\Delta$sensitivity \\
%    \hline
%    tRNA & 557 & 74 & 77.3 & 67.73 & 69.12 & +0.00 & +0.00 & 61.75 & 70.98 & +0.04 & -0.07 \\
%    5S rRNA & 1,283 & 1,125 & 118.8 & 72.60 & 72.59 & -0.01 & +0.00 & 57.28 & 63.35 & +0.00 & +0.05 \\
%    SRP  & 928 & 886 & 186.1 & 59.67 & 59.02 & -0.04 & -0.03 & 56.58 & 61.55 & -0.07 & -0.15 \\
%    RNaseP & 454 & 182 & 344.1 & 47.49 & 46.15 & -0.18 & -0.19 & 45.76 & 53.28 & +0.07 & -0.05 \\
%    tmRNA  & 462 & 462 & 366.0 & 43.83 & 37.38 & (*)-0.68 & (*)-0.70 & 39.75 & 44.90 & (*)-1.08 & (*)-1.15 \\
%    Group I Intron & 98 & 96 & 424.9 & 51.03 & 49.35 & (**)+1.15 & (*)+0.89 & 45.49 & 56.06 & (*)+0.81 & (*)+0.97 \\
%    telomerase RNA & 37 & 37 & 444.6 & 43.66 & 56.72 & +0.04 & +0.09 & 39.53 & 55.40 & -0.05 & -0.23 \\
%    16S rRNA & 22 & 22 & 1,547.9 & 39.01 & 39.97 & (*)+3.73 & (*)+2.95 & 35.65 & 42.26 & (*)+1.24 & (*)+1.29 \\
%    23S rRNA & 5 & 5 & 2,927.4 & 46.46 & 47.97 & (**)+13.42 & (**)+9.52 & 53.20 & 60.50 & +0.13 & -0.05 \\
%    \hline
%    {\em Overall} & 3,846 & 2,889 & 222.2 & 52.39 & 53.14 & +1.95 & +1.38 & 48.33 & 56.48 & +0.12 & +0.06 \\
%\end{tabular}
%\smallskip
%\caption{The prediction accuracies using exact base-pair matching. Statistical significance are marked by 
%         *($0.01\leq p<0.05$) and **($p<0.01$).
%                  Overall, \linearfoldc outperforms \contrafoldmfe by +1.95 PPV and +1.38 sensitivity,
%                  and \linearfoldv outperforms \viennarnafold by +0.12 PPV and +0.06 sensitivity.
%         Among the nine families, \linearfoldc is significantly better on three (Group I Intron, 16S and 23S rRNAs),
%               and \linearfoldv is significantly better on two (Group I Intron and 16S rRNAs).
%\label{tab:accuracy_nos}}
%\end{table*}


\begin{table*}[h]  % hzhang: redo with new data
  \centering
  \begin{tabular}{r|rrr||rr|rr||rr|rr}
    & \multicolumn{2}{c}{\# of sequences} & avg. & \multicolumn{2}{c|}{\contrafoldmfe} & \multicolumn{2}{c||}{\linearfoldc $b$=100} & \multicolumn{2}{c|}{\viennarnafold} & \multicolumn{2}{c}{\linearfoldv $b$=100}\\
    Family & total & used & length & PPV & sensitivity & $\Delta$PPV & $\Delta$sensitivity & PPV & sensitivity & $\Delta$PPV & $\Delta$sensitivity \\
    \hline
    tRNA & 557 & 74 & 77.3 &  67.61 & 69.12 & +0.00 & +0.00 & 61.75 & 70.98 & +0.04 & -0.07 \\
    5S rRNA & 1,283 & 1,125 & 118.8 &  70.68 & 70.70 & +0.00 & +0.00 & 57.28 & 63.35 & -0.14 & -0.11 \\
    SRP  & 928 & 886 & 186.1 & 59.14 & 58.61 & -0.05 & -0.07 & 56.58 & 61.55 & -0.09 & -0.20 \\
    RNaseP & 454 & 182 & 344.1 &  47.45 & 46.39 & -0.25 & (*)-0.55 & 45.76 & 53.28 & +0.15 & +0.04 \\
    tmRNA  & 462 & 462 & 366.0 &  42.96 & 36.94 & (*)-0.81 & (**)-0.99 & 39.75 & 44.90 & (**)-1.09 & (**)-1.17  \\
    Group I Intron & 98 & 96 & 424.9 &  51.21 & 49.56 & +0.80 & (*)+0.75 & 45.49 & 56.06 & (**)+0.81 &(*)+0.97  \\
    telomerase RNA & 37 & 37 & 444.6 & 43.40 & 56.58 & +0.03 & +0.00 & 39.53 & 55.40 & -0.05 & -0.19 \\
    16S rRNA & 22 & 22 & 1,547.9 &  39.84 & 40.49 & (*)+3.47 & (*)+2.99 & 35.65 & 42.26 & (*)+1.33 & +1.39 \\
    23S rRNA & 5 & 5 & 2,927.4 & 50.56 & 51.24 & (*)+8.51 & (*)+5.60 & 53.20 & 60.50 & +0.07 & -0.12  \\
    \hline
    {\em Overall} & 3,846 & 2,889 & 222.2 &52.54  &53.29  & +1.30 &+0.86  &48.33  &56.48  &+0.11  &+0.06 \\
  \end{tabular}
  \smallskip
  \caption{The prediction accuracies using exact base-pair matching. Statistical significance are marked by 
    *($0.01\leq p<0.05$) and **($p<0.01$).
    Overall, \linearfoldc outperforms \contrafoldmfe by +1.80 PPV and +1.23 sensitivity, % hzhang: redo with new data
    and \linearfoldv outperforms \viennarnafold by +0.11 PPV and +0.06 sensitivity. % hzhang: redo with new data
    Among the nine families, \linearfoldc is significantly better on five (SRP, Group I Intron, 5S, 16S and 23S rRNAs), % hzhang: grp1, 16s and 23s are significant only on PPV
    and \linearfoldv is significantly better on two (Group I Intron and 16S rRNAs).  % hzhang: 16s is significant only on PPV
    \label{tab:accuracy_nos}}
\end{table*}
    
%\newpage

\begin{figure}% [h]
  \centering
%\hspace{-0.5cm}
\begin{tabular}{ll}
{\large\sf A} & {\large\sf B}\\[-0.5cm]
    \includegraphics[width=.45\textwidth]{figs/beam_PPV_Sens_vienna_Mathews_hzhang.pdf}
    &
    \hspace{-1cm}
    \includegraphics[width=.45\textwidth]{figs/precision-recall-Mathews-vienna-hzhang.pdf}
  \end{tabular}
  \caption{This Figure corresponds to Figure~\ref{fig:beamsize}C--D but
  using the \viennarnafold instead of \contrafoldmfe model, on the same ArchiveII dataset. 
    A: PPV and sensitivity against beam size;
    B: sensitivity against PPV, with varying beam size.
    Similar to what we observe in Figure~\ref{fig:beamsize},
    PPV and sensitivity both improve with beam size, up to 120 or so,
    and then slowly degrade, eventually converging to the exact search baseline of \viennarnafold.
    The best beam size appears to be around 120, but we choose the round number of 100 by default.
    We do not fine-tune the beam size on any dataset.
  \label{fig:viennapr}}
%\vspace{1cm}
\end{figure}



\begin{figure}
\centering
\includegraphics[width=0.4\textwidth]{figs/beamsize_runtime_23s_hzhang.pdf}
\caption{Impact of beam size on prediction speed (results on 23S rRNAs only); note the x-axis is beam size, not sequence length.
The running times of \linearfold without cube pruning scale superlinearly  with beam size $b$, which agrees with the theoretical analysis of $O(nb^2)$ time,
and linearly with beam size after applying cube pruning, which also agrees with the theoretical analysis of $O(nb\log b)$ time.
All other results in this paper use cube pruning.
\label{fig:beamsize_speed}
}
\end{figure}

%% \iffalse
%% \begin{figure} %[h]
%%   \centering
%%   \begin{tabular}{cc}
%%       \includegraphics[width=.4\textwidth]{figs/runtime_ppv.pdf}
%%     &\hspace{-1cm}
%%       \includegraphics[width=.4\textwidth]{figs/runtime_sens.pdf}
%%       \\[-0.2cm]
%%   \end{tabular}
%%   \caption{The PPV/sensitivity against running time, comparing our \linearfold
%%     with two baseline systems, running on the ArchiveII dataset. 
%%     \label{fig:runningtime} }
%% \end{figure}
%% \fi

\begin{figure} %[h]
  \centering
  \begin{tabular}{cc}
      \includegraphics[width=.35\textwidth]{figs/bylen_bar_v_precision_hzhang.pdf} %hzhang: replot with new data
    &\hspace{-1.2cm}
      \includegraphics[width=.35\textwidth]{figs/bylen_bar_v_recall_hzhang.pdf} %hzhang: replot with new data
      \\[-0.2cm]
  \end{tabular}
  \caption{This Figure corresponds to Figure~\ref{fig:accuracy}C, with \viennarna instead of \contrafoldmfe model, on the same ArchiveII dataset.
    We also include the local folding results of \viennarnafold with maximum base pair span of 150 (using {\tt --maxBPspan 150} in \rnafold).
%    It shows PPV and sensitivity of \linearfoldv by pair length
%    in the Mathews
%    dataset, comparing to \viennarnafold.
    Each bar represents the overall PPV/sensitivity of all base pairs in a particular length range (by groundtruth pairs).
    We observe that \linearfoldv outperforms \viennarnafold in long-range base pairs (500+ \nts apart),
       while the local folding method only predicts local pairs ($\leq$150 \nts apart).
    \label{fig:viennabylen} }
\end{figure}

\begin{figure} %[h]
  \centering
  \begin{tabular}{ll}
  {\large\sf A} & \hspace{-0.4cm}{\large\sf B}
             \\[-0.5cm]
      \ \ \includegraphics[width=.5\textwidth]{figs/bylen_bracnt_hzhang.pdf}
    &\hspace{-.2cm}
      \includegraphics[width=.5\textwidth]{figs/bylen_bracnt_v_hzhang.pdf}
      \\[-0.2cm]
  \end{tabular}
  \caption{Length distributions of the predicted base pairs using different methods, on the
    16S/23S rRNAs in the ArchiveII dataset.
    Here we plot the number of both predicted and ground truth base pairs (including pseudoknots)
    in each of the following ranges:
    $(0,50], (50,100]$, ... $(400,500]$.
    This figure shows that \linearfoldc  %(with both beam sizes 100 and 50)
    produces almost the same length distributions with the ground truth,
    while \contrafold severely overpredicts base pairs longer than 500\nts apart.
    Both \viennarna and \linearfoldv overpredict in that range, but \linearfoldv is less severe.
    %% has almost the same 
    %% is very consistent with the
    %% distribution of the base pair predicted by length, even with a very small
    %% number of beamsize.
    In B, we also reconfirm the limitation of local folding which does not output any long-range pairs.
    %% algorithm can
    %% predict no base pair beyond their maximum range, which is a crucial drawback
    %% in prediction of long RNA sequences. 
    \label{fig:bylencnt}}
\end{figure}


\definecolor{LightCyan}{rgb}{0.88,1,1}
\begin{figure*} [t]
  \newcolumntype{g}{>{\columncolor{LightCyan}}c}
  \centering
  \hspace{-0.3cm}
  \begin{tabular}{|c|c|g|}
\hline%\hhline{~|-|-|-|}%\cline{2-4}
 \multicolumn{3}{|c|}{23S rRNA \ecoli (2904 \nts, 830 pairs)}\\
% \hline%\hhline{~|-|-|-|}%\cline{2-4}
%    \contrafoldmfe   %Group I Intron \cmirabilis (526\nts, 143 \pairs)
%     &
%       \contrafoldmfe Local ($L$=150)  % 16S rRNA \apyrophilus (1564\nts, 468 \pairs)
%     &
%       \linearfoldc $b$ = 100 % 23S rRNA \ecoli (2904\nts, 830 \pairs)
%     \\

% \hline%\hhline{~|-|-|-|}%\cline{2-4}

%     \includegraphics[width=.3\textwidth]{figs/23s_4_contrafold.pdf}
%     & 
%       \includegraphics[width=.3\textwidth]{figs/23s_4_contrafold150.pdf}
%     &
%       \includegraphics[width=.3\textwidth]{figs/23s_4_beam100.pdf}
%     \\
    
%       {\scriptsize \ppv: 47.15, \sens: 48.92, \pairs: 861}%, time: 5.20$s$. }
%     &
%       {\scriptsize \ppv: 63.70, \sens: 54.34, \pairs: 708}%, time: 5.20$s^*$. }
%     &
%       {\scriptsize \ppv: 69.94, \sens: 67.83, \pairs: 805}%, time: 1.78$s$. }
%     \\

\hline%\hhline{~|-|-|-|}%\cline{2-4}
    \viennarnafold   %Group I Intron \cmirabilis (526\nts, 143 \pairs)
    &
      \viennarnafold Local ($L$=150)  % 16S rRNA \apyrophilus (1564\nts, 468 \pairs)
    &
      \linearfoldv $b$ = 100 % 23S rRNA \ecoli (2904\nts, 830 \pairs)
    \\

\hline%\hhline{~|-|-|-|}%\cline{2-4}
    
      \includegraphics[width=.32\textwidth]{figs/23s_4_vienna_hzhang.pdf}
    & 
      \includegraphics[width=.32\textwidth]{figs/23s_4_vienna150_hzhang.pdf}
    &
      \includegraphics[width=.32\textwidth]{figs/23s_4_linearvienna_hzhang.pdf}
    \\

      {\scriptsize \ppv: 52.02, \sens: 57.47, \pairs: 917, time: 7.6$s$. }
    &
      {\scriptsize \ppv: 51.63, \sens: 55.18, \pairs: 887, time: 6.3$s$. }
    &
       {\scriptsize \ppv: 55.60, \sens: 62.17, \pairs: 928, time: 2.2$s$. }
    \\
    \hline
    
% &    Group I Intron \cmirabilis (526\nts, 143 \pairs)
%     &
%       16S rRNA \apyrophilus (1564\nts, 468 \pairs)
%     &
%       23S rRNA \ecoli (2904\nts, 830 \pairs)
%     \\
%     \hline
% \rot{\hspace{1.5cm}\viennarnafold}    
%  &   \includegraphics[width=.3\textwidth]{figs/new_4_vienna.pdf}
%     &
%     \includegraphics[width=.3\textwidth]{figs/16s_8_vienna.pdf}
%     &
%     \includegraphics[width=.3\textwidth]{figs/23s_4_vienna.pdf}
%     % \end{subfigure}

%     \\

%  &   \ppv: 45.61, \sens: 54.55, \pairs: 171.
%     &
%     \ppv: 56.12, \sens: 63.68, \pairs: 531.
%     &
%     \ppv: 52.34, \sens: 57.83, \pairs: 917.
%     \\[0.3cm]
% \hdashline

% %    \hline %%%%%%%%%%%%%%%%%%%%%%%%%%%%%%%%%%
    
% \rot{\hspace{1.5cm}\linearfoldv $b$=100}    
%  &   \includegraphics[width=.3\textwidth]{figs/new_4_linearvienna.pdf}
%     &
%     \includegraphics[width=.3\textwidth]{figs/16s_8_linearvienna.pdf}
%     &
%     \includegraphics[width=.3\textwidth]{figs/23s_4_linearvienna.pdf}
%     \\

%   &  \ppv: 56.63, \sens: 65.73, \pairs: 166.
%     &
%     \ppv: 55.81, \sens: 62.61, \pairs: 525.
%     &
%     \ppv: 56.14, \sens: 62.77, \pairs: 928.
%     \\

  \end{tabular}
  \caption{Circular plots of prediction results on the \ecoli 23S rRNA  (corresponding to
    Figure~\ref{fig:circular}) comparing \viennarnafold, its local folding mode, and \linearfoldv.
    We observe that local folding only predicts local pairs ($\leq$150 \nts apart in this example).
    Moreover, with a time complexity of $O(nL^2)$, it runs almost as slow as the $O(n^3)$ baseline, and much slower than our \linearfold (with complexity $O(nb\log b)$).
    %, and local folding method using the same model.
    \label{fig:si-circle}}
\end{figure*}


\begin{figure} %[h]
  \centering
  \begin{tabular}{ll}
  {\large\sf A} & {\large\sf B}\\[-0.5cm]
    \includegraphics[width=.5\textwidth]{figs/modelscore_fscore_Mathews_hzhang.pdf}
    &\hspace{-.2cm}
      \includegraphics[width=.5\textwidth]{figs/modelscore_fscore_Mathews_16s23s_hzhang.pdf}
      \\[-0.2cm]
  \end{tabular}
  \caption{Comparing left-to-right and right-to-left prediction,
    using the whole Archive II dataset (A) and only 16S/23S families (B).
    In general, they perform very similarly.
    %% We run a set of experiment to see if our algorithm reflects the
    %% effects of cotrancriptional folding, by comparing predictions from
    %% left-to-right and right-to-left. Experimental results show that 
    %% left-to-right and right-to-left prediction perform similarly in terms of PPV
    %% and sensitivity (left). Specifically, in 16S/23S families, right-to-left prediction
    %% is actually performing slightly better (right). 
    \label{fig:right to left} }
\end{figure}

\begin{figure}[h]
  \vspace{-0.1cm}
  \hspace{-.5cm}
  \centering
  \scalebox{1.2}{
    \begin{tabular}{c}
      \begin{tabular}{ll}
        
  \begin{tabular}{ll}
   input      & $x_1 \ldots x_{n}$ \\[0.1in]
%\hline\\
    state        & \nnitem{\bm y}{\sigma}{j}{s}  \\ [0.1in]

    axiom        & \nnitem{\bm {\text `}~{\text '}}{{\text `}~{\text '}}{1}{0} \\ [0.1in]

    goal         & \nnitem{\bm y}{{\text `}~{\text '}}{n+1}{\_}  \\ [0.1in]

    {\push}       & \inferrule{\nnitem{\bm y}{\sigma}{j}{s}}
                              {\nnitem{\bm y\!\circ\!{\text `}\leftb{\text '}}{\sigma \newmid j}{j\!+\!1}{0} }   \\[0.2in]

    {\nskip}       & \inferrule{\nnitem{\bm y}{\sigma}{j}{s}}
                               {\nnitem{\bm y\!\circ\!{\text `}\mydot{\text '}}{\sigma}{j\!+\!1}{s+\delta} }   \\[0.2in]

    {\pop}     & \inferrule{\nnitem{\bm y}{\sigma \newmid i}{j}{s}}
                           {\nnitem{\bm y\!\circ\!{\text `}\rightb{\text '}}{\sigma}{j\!+\!1}{s+\xi_{ij}}}
                                                \hspace{.1cm} $~(x_i, x_j)~$ match  
    \\[0.1in]
  \end{tabular}
&
%\hspace{-.6cm}  
  \begin{tabular}{ll}
   input      & $x_1 \ldots x_n$ \\[0.1in]
%\hline\\
    state        & \nitemt{\bm \sigma}{j}{\bm{y}, s}  \\ [0.1in]

    axiom        & \nitemt{\bm ~{\text `}~{\text '}}{1}{{\text `}{\text '}, 0} \\ [0.1in]

    goal         & \nitemt{\bm ~{\text `}~{\text '}}{n+1}{\bm{y}, \_}  \\ [0.1in]

    {\push}       & \inferrule{\nitemt{\bm \sigma}{j}{\bm{y},s}}
                              {\nitemt{{\bm \sigma} \newmid j}{j\!+\!1}{\bm{y}\!\circ\!{\text `}\leftb{\text '},0}}   \\[0.2in]

    {\nskip}       & \inferrule{\nitemt{\bm \sigma}{j}{\bm{y}, s}}
                               {\nitemt{\bm \sigma}{j\!+\!1}{\bm{y}\!\circ\!{\text `}\mydot{\text '}, s+\delta}}   \\[0.2in]

    {\pop}     & \inferrule{\nitemt{{\bm \sigma} \newmid i}{j}{\bm{y},s}}
                           {\nitemt{\bm \sigma}{j\!+\!1}{\bm{y}\!\circ\!{\text `}\rightb{\text '}, s+\xi_{ij}}}%{A\cup \{(i,j)\}}}  
                                                \hspace{.1cm} $~(x_i, x_j)~$ match  
    \\[0.1in]
  \end{tabular}
   \\
              & \\
  & \\
  ~~~~(a) exhaustive, $O(3^n)$ & ~~~~(b) Idea 1: identical-stack merge , $O(2^n)$ \\
              & \\
  & \\
\end{tabular}

      \\
      
\begin{tabular}{lll}
%   input:      & 
input & $x_1 \ldots x_n$ & \codeblue{$\alpha$} denotes a bracket-closed substructure  \\[0.1in]
%\hline\\
%    state:        & 
state & \nitemt{i}{j}{\bml\codeblue{$\alpha$},s} & a dummy symbol \bml is added at the beginning of the state \\[0.1in]%\ensuremath{{\bm y}: \raisebox{-0.25em}{\tiny $i$} \bml \codeblue{\ensuremath{\alpha}} \raisebox{-0.25em}{\tiny $j$}} ~or~ \ensuremath{\raisebox{-0.25em}{\tiny $0$} \codeblue{\ensuremath{\alpha}} \raisebox{-0.25em}{\tiny $j$}}\;\\ [0.1in]

% axiom:        & 
axiom & \nitemt{0}{1}{\bml,0} \\ [0.1in] % & \ensuremath{{\text `}~{\text '}: \raisebox{-0.25em}{\tiny $0$}\codeblue{\!} \raisebox{-0.25em}{\tiny $1$}}\\ [0.1in]

%    goal:         & 
goal & \nitemt{0}{n+1}{\bml\codeblue{$\alpha$},\_} \\ [0.2in] % & \ensuremath{{\bm y}: \raisebox{-0.25em}{\tiny $0$} \codeblue{\ensuremath{\alpha}} \raisebox{-0.25em}{\tiny $n+1$}}  \\ [0.2in]

%    {\push}       & 
{\push} & \inferrule{\nitemt{i}{j}{\bml\codeblue{$\alpha$},s}}
                              {\nitemt{j}{j\!+\!1}{\bml,0}} & \\ [0.2in]%\inferrule{{\bm y}: \raisebox{-0.25em}{\tiny $i$}\bml \codeblue{\ensuremath{\alpha}} \raisebox{-0.25em}{\tiny ~$j$}}{\quad\quad\quad\quad\quad\quad\raisebox{-0.25em}{\tiny $j$}\bml \codeblue{\!} \raisebox{-0.25em}{\tiny ~$j\Plus 1$}} \\ [0.2in]

%    {\nskip}       & 
{\nskip} & \inferrule{\nitemt{i}{j}{\bml\codeblue{$\alpha$},s}}
                               {\nitemt{i}{j\!+\!1}{\bml\codeblue{$\alpha\md$},s+\delta}} \\ [0.2in]% & \inferrule{{\bm y}: \raisebox{-0.25em}{\tiny $i$}\bml \codeblue{\ensuremath{\alpha}} \raisebox{-0.25em}{\tiny ~$j$}}{\quad\quad\quad\raisebox{-0.25em}{\tiny $i$}\bml \codeblue{\ensuremath{\alpha\mydot}} \raisebox{-0.25em}{\tiny ~$j\Plus 1$}~} \\ [0.2in]

%    {\pop}     & 
{\pop} & \inferrule{\nitemt{k}{i}{\bml\codeblue{$\alpha$},s'} \quad \nitemt{i}{j}{\bml\codeblue{$\beta$},s}}
                           {\nitemt{k}{j\!+\!1}{\bml\codeblue{$\alpha\ml\beta\mr$} ,s'+s+\xi_{ij}}} 
         & $~(x_i, x_j)~$ match 
%         \;\; \inferrule{{\bm y'}: \raisebox{-0.25em}{\tiny $k$} $\bml \codeblue{\ensuremath{\alpha}}$ \raisebox{-0.25em}{\tiny ~$i$}\quad\quad\quad{\bm y}: \raisebox{-0.25em}{\tiny $i$}\bml \codeblue{\ensuremath{\beta}} \raisebox{-0.25em}{\tiny ~$j$}~}{\raisebox{-0.25em}{\tiny $k$}\bml \codeblue{\ensuremath{\alpha~\ml~\beta~\mr}} \raisebox{-0.25em}{\tiny ~$j\Plus 1$}\quad\quad}
  \\ [0.2in]
  % {\ncombine} & \inferrule{\nitemt{0}{i}{\bml\codeblue{$\alpha$},s'} \quad \nitemt{i}{j}{\bml\codeblue{$\beta$},s}}
  %               {\nitemt{0}{j}{\bml\codeblue{$\alpha\,\beta$}, s'+s} } \\
  % & \inferrule{{\bm y'}: \raisebox{-0.25em}{\tiny $0$}\codeblue{\ensuremath{\alpha}} \raisebox{-0.25em}{\tiny $i$}\quad\quad{\bm y}: \raisebox{-0.25em}{\tiny $i$} \codeblue{\ensuremath{\beta}} \raisebox{-0.25em}{\tiny ~$j$}~}{\quad\raisebox{-0.25em}{\tiny 0}\codeblue{\ensuremath{\alpha~\beta}} \raisebox{-0.25em}{\tiny ~$j$}\;\quad\quad\quad\quad} 
%    \\
              & \\
                & \\
  \multicolumn{3}{c}{~~~~(c) Idea 2: stack-top packing, $O(n^3)$}
\end{tabular}
\end{tabular}
  }
\caption{The simplified deductive system of (a) exhaustive incremental search,
  (b) merging by identical-stack, and (c) packing by stack-top as in \linearfold.
  Here $\circ$ denotes string concatenation; $\delta$ scores the unpaired
  nucleotide, and $\xi_{ij}$ scores a base pair. 
We say $(a, b)$ ``matches'' if $(a,b)$ is one of the allowed pairs ({\tt C--G}, {\tt A--U}, {\tt G--U}).
We also explain the meaning of the deductive system in (c) by highlighting bracket-closed substructures,
i.e., a substring with balanced parentheses. % , e.g., {\tt ..}, {\tt (..)}, {\tt .().}, etc.
\label{fig:deduction}}
\end{figure}

\iffalse
\begin{figure}
  \centering
  \tikzset{>={stealth[scale=0.01]}}
  \newcommand{\DrawNode}[2][]{%
    \draw  [#1]
    ([line width=0.02mm, shift={(-0.5\pgflinewidth,-0.5\pgflinewidth)}]#2.north east) -- 
    ([line width=0.02mm, shift={(0.5\pgflinewidth,-0.5\pgflinewidth)}]#2.north west) -- 
    ([line width=0.02mm, shift={(0.5\pgflinewidth,+0.5\pgflinewidth)}]#2.south west) -- 
    ([line width=0.02mm, shift={(-0.5\pgflinewidth,+0.5\pgflinewidth)}]#2.south east)
  }%
% \hspace{2em}
\begin{tabular}{c}
  \setlength\tabcolsep{0.0em}  
  \begin{tikzpicture}[
        scale=1.5,
    every node/.style={scale=1.5},
    tcancel/.append style={draw=#1, cross out, inner sep=1pt},
    ]
    \matrix [matrix of nodes, row sep=0.07em, column sep=2.0em, align=left, nodes={inner sep=0.2em}] { %
      &              &                & [-1.0em]    &              & \\
       & {\tt step i} & {\tt step i+1} &    {\tt ...}     & {\tt step j} & {\tt step j+1} \\
      \node(p00)[] {}; & \node(p01)[draw, scale = 1.2, minimum width = 6em] {\mq\mq$_{k_1}$\bml\codeblue{$\alpha_1$}~\color{white}{~~}}; & & & & \node(p05)[draw, scale = 1.2, minimum width = 6.5em] {\mq\mq$_{k_1}$\bml\codeblue{$\alpha_1$\ml$\beta$\mr}~\color{white}{~~}};\\
      & \node(Ap01)[scale=1.00, inner sep=0.1em] {\small .. $k_1$\color{white}{}}; & & & & \node(Ap05)[scale=1.00, inner sep=0.1em] {\small .. $k_1$\color{white}{}}; \\ [1em]
      \node(p10)[] {}; & \node(p11)[draw, scale = 1.2, minimum width = 6em] {\mq$_{k_2}$\bml\codeblue{$\alpha_2$}~\color{white}{~~~~}}; & \node(p12)[draw, scale = 1.2] {\mq\mq$_{i}$\bml\codeblue{~}}; & & \node(p14)[draw, scale = 1.2] {\mq\mq$_{i}$\bml\codeblue{$\beta$}~}; & \node(p15)[draw, scale = 1.2, minimum width = 6.5em] {\mq$_{k_2}$\bml\codeblue{$\alpha_3$\ml$\beta$\mr}~\color{white}{~~~~}};\\
      & \node(Ap11)[scale=1.00, inner sep=0.1em] {\small .. $k_2$\color{white}{}}; & \node(Ap12)[scale=1.00, inner sep=0.1em] {\small .. $i$\color{white}{}};& & \node(Ap14)[scale=1.00, inner sep=0.1em] {\small .. $i$\color{white}{}}; & \node(Ap15)[scale=1.00, inner sep=0.1em] {\small .. $k_2$\color{white}{}};\\ [1em]
        \node(p20)[] {}; & \node(p21)[draw, scale = 1.2, minimum width = 6em] {\mq\mq\mq$_{k_3}$\bml\codeblue{$\alpha_3$}~\color{white}{}}; & & & & \node(p25)[draw, scale = 1.2, minimum width = 6.5em] {\mq\mq\mq$_{k_3}$\bml\codeblue{$\alpha_3$\ml$\beta$\mr}~\color{white}{}}; \\
        & \node(Ap21)[scale=1.00, inner sep=0.1em] {\small .. $k_3$\color{white}{}}; & & & & \node(Ap25)[scale=1.00, inner sep=0.1em] {\small .. $k_3$\color{white}{}};\\ [1em]
      };
      \DrawNode{Ap01};\DrawNode{Ap11};\DrawNode{Ap21};
      \DrawNode{Ap12};
      \DrawNode{Ap14};
      \DrawNode{Ap05};\DrawNode{Ap15};\DrawNode{Ap25};

      \draw[->, thick, color=red] (p01.0) -- node (loc4) [fill=white, inner sep=1pt, midway] {\scriptsize \ml} (p12.180); 
      \draw[->, thick, color=red] (p11.0) -- node [fill=white, inner sep=1pt, midway] {\scriptsize \ml} (p12.180); 
      \draw[->, thick, color=red] (p21.0) -- node (loc2) [fill=white, inner sep=1pt, midway] {\scriptsize \ml} (p12.180); 
      
      \draw[->, thick] (p12.0) -- node (locstar) [fill=white, inner sep=1pt, midway] {\scriptsize .~.~.} (p14.180);       
      
      \draw[->, thick, color=green] (p14.0) -- node (loc5) [fill=white, inner sep=1pt, midway] {\scriptsize \mr} (p05.180); 
      \draw[->, thick, color=green] (p14.0) -- node [fill=white, inner sep=1pt, midway] {\scriptsize \mr} (p15.180); 
      \draw[->, thick, color=green] (p14.0) -- node (loc3) [fill=white, inner sep=1pt, midway] {\scriptsize \mr} (p25.180); 
      
      % \draw[draw=blue!60,line width=3mm,-{Latex[length=6mm]},shorten >=1mm, shorten <=1mm] ($(loc3.270)+(0,-.15)$) -- ($(loc2.270)+(0,-.15)$)
      % node[midway,text=white,font=\tiny,sloped]{unpack from packing};
      % \draw[draw=blue!60,line width=3mm,-{Latex[length=6mm]},shorten >=1mm, shorten <=1mm] ($(loc5.90)+(0,0)$) -- ($(loc4.90)+(0,0)$)
      % node[midway,text=white,font=\tiny,sloped]{unpack from packing};
      % \node [above=.3em of locstar] {\scriptsize recursive}; 
    \end{tikzpicture}
\\ \end{tabular}
    
  \caption{ A demonstration of the packing and unpacking process in
    % Figure~\ref{fig:deduction} (c)(d).
    stack-top dynamic programming. 
    \label{fig:packing} }
\end{figure}
\fi

\begin{figure}
  \centering
  \includegraphics[width=.9\textwidth]{figs/pseudocode_deductivesystem.pdf}
  \caption{Pseudocode of a simplified version of our \linearfold algorithm,
    corresponding to Figure~\ref{fig:deduction}(c) plus beam search.
    For the whole \linearfold system, please refer to our source code on Github.
    Here we model hash tables following Python dictionaries,
    where $(i,j) \in c$ checks whether the key $(i,j)$ is in the hash $c$; this is needed to ensure linear runtime.
    \label{fig:pseudocode}}
\end{figure}

\newcommand{\ShapeE}{\ensuremath{\mathsf{E}}}
\newcommand{\ShapeP}{\ensuremath{\mathsf{P}}}
\newcommand{\ShapeH}{\ensuremath{\mathsf{H}}}
\newcommand{\ShapeM}{\ensuremath{\mathsf{M_1}}}
\newcommand{\ShapeMt}{\ensuremath{\mathsf{M_2}}}
\newcommand{\ShapeMulti}{\ensuremath{\mathsf{M}}}
\newcommand{\hjump}{\ensuremath{\ShapeH\mathsf{jump}}\xspace}
\newcommand{\npair}{\ensuremath{\mathsf{hairpin}}\xspace}
\newcommand{\nreduce}{\ensuremath{\mathsf{reduce}}\xspace}
\newcommand{\single}{\ensuremath{\mathsf{singleloop}}\xspace}
\newcommand{\multi}{\ensuremath{\mathsf{multiloop}}\xspace}
\newcommand{\multileft}{\ensuremath{\ShapeMulti\mathsf{left}}\xspace}
\newcommand{\multijump}{\ensuremath{\ShapeMulti\mathsf{jump}}\xspace}

\newcommand{\xtom}{\ensuremath{\mathsf{XtoM_1}}\xspace}
\newcommand{\myus}{\ensuremath{\text{\tt \_\,}}\xspace}

\begin{figure}
  \centering
  \scalebox{1.2}{
    \begin{tabular}{lll}
      input      & $x_1 \ldots x_n$ &  \\[0.1in]
      states & $\ShapeE$ \nitemt{0}{j}{\codeblue{$\alpha$}, s}  &  prefix structure \\[0.1in] % & $_0 \codeblue{\ensuremath{\myus\myus\myus}}_j$~, where \codeblue{\ensuremath{\myus\myus\myus}} denotes a bracket-closed substructure \\[0.1in]
                 & $\ShapeP$ \nitemt{i}{j}{\codeblue{\ml$\alpha$\mr}, s} & pair \\[0.1in] % & $_i\codeblue{\ensuremath{\leftb\myus\myus\myus\rightb}}_j$ \\[0.1in]
                 & $\ShapeH$ \nitemt{i}{j}{\bml\codeblue{\md\md\md},s} & hairpin candidate \\[0.1in] % & $_i\bml\codeblue{\ensuremath{\md\md\md}}_j$~,  where \codeblue{\ensuremath{\md\md\md}} is 1+ unpaired nucleotide(s) \\[0.1in]
                 & $\ShapeM$ \nitemt{i}{j}{\codeblue{\ml$\alpha$\mr$\beta$},s} & one or more pairs \\[0.1in]%  & $_i \underbracket[0.1em]{\overbracket[0.1em]{\codeblue{\ensuremath{\leftb\myus\myus\myus\rightb}}}^{\ShapeP}\!\!\codeblue{\ensuremath{\md\md\md~\myus\myus\myus}}\!\!\overbracket[0.1em]{\codeblue{\ensuremath{\leftb\myus\myus\myus\rightb}}}^{\ShapeP}\!\codeblue{\ensuremath{\md\md\md}}}_{\text{
                                                                                                                   % 1+ \ShapeP s, each followed by 0+ unpaired nucs
                 %followed by 0+ unpaired nucs
                                                                                                                   % }} \;_j$ \\[0.2in]
                 & $\ShapeMt$ \nitemt{i}{j}{\codeblue{\ml$\alpha$\mr$\beta$\ml$\gamma$\mr},s} & two or more pairs \\[0.1in] %  & $_i \underbracket[0.1em]{\overbracket[0.1em]{\codeblue{\ensuremath{\leftb\myus\myus\myus\rightb}}}^{\ShapeP}\!\!\codeblue{\ensuremath{\md\md\md~\myus\myus\myus}}\!\!\overbracket[0.1em]{\codeblue{\ensuremath{\leftb\myus\myus\myus\rightb}}}^{\ShapeP}}_{\text{2+ \ShapeP s, separated by 0+ unpaired nucs
                                                                                                                                       % }} \;_j$ \\[0.2in]
                 & $\ShapeMulti$ \nitemt{i}{j}{\bml\codeblue{\md\md\md\ml$\alpha$\mr$\beta$\ml$\gamma$\mr\md\md\md},s} &  multiloop candidate \\[0.2in] % & $_i \underbracket[0.1em]{\bml\codeblue{\ensuremath{\md\md\md}}\!\!\overbracket[0.1em]{\codeblue{\ensuremath{\leftb\myus\myus\myus\rightb}}}^{\ShapeP}\!\!\codeblue{\ensuremath{\md\md\md~\myus\myus\myus}}\!\!\overbracket[0.1em]{\codeblue{\ensuremath{\leftb\myus\myus\myus\rightb}}}^{\ShapeP}\!\codeblue{\ensuremath{\md\md\md}}}_{\text{
                 % a \bml followed by 2+~\ShapeP s, %separated by 0+ unpaired nucs
                 % each surrounded by 0+ unpaired nucs  %(multiloop candidate)
                 % }} \;_j$ \\[0.3in]
                 
                 %$_i \underbracket[0.1em]{\overbracket[0.1em]{\codeblue{\ensuremath{\leftb\myus\myus\myus\rightb}}}^{\ShapeP}\!\!\codeblue{\ensuremath{\md\md\md~ \cdot\cdot\cdot\cdot\cdot\cdot}}\!\!\overbracket[0.1em]{\codeblue{\ensuremath{\leftb\myus\myus\myus\rightb}}}^{\ShapeP}}_{\text{two or more~} \ShapeP\text{s}} \;_j$ \\[0.1in]
      axiom & $\ShapeE$ \nitemt{0}{1}{\codeblue{\!},0} %\\[0.1in]
      & goal \qquad $\ShapeE$ \nitemt{0}{n\!\!+\!1}{\codeblue{$\alpha$},\_} \\[0.1in]
      {\push}  & \inferrule{\ \ \ \ \ \ \ \ \ \  \ShapeE \nitemt{0}{j}{\codeblue{$\alpha$},s}}{ \ShapeH \nitemt{j}{\jnext(j,j)}{\bml\codeblue{\md\md}, 0}} & $\jnext(i,j) \triangleq \min\{k \mid k>j, \ (x_i,x_{k})~\text{match}\}$ \\[0.2in] %  
      {\hjump} & \inferrule{\ \ \ \ \ \ \ \ \ \ \ShapeH \nitemt{i}{j}{\bml\codeblue{\md\md\md}, s}}{ \ShapeH \nitemt{i}{\jnext(i,j)}{\bml\codeblue{\md\md\md\md\md}, s}} \\[0.2in]
      {\nskip}   & \inferrule{ \ShapeE \nitemt{0}{j}{\codeblue{$\alpha$}, s} \ \ \ \ \ \ }{ \ShapeE \nitemt{0}{j\!\!+\!1}{\codeblue{$\alpha$\md}, s \!\!+\! \delta^E}} %\\[0.2in]
                 & \inferrule{ \ShapeM \nitemt{i}{j}{\codeblue{$\ml\alpha\mr\beta$}, s} \ \ \ \ \ \ }{ \ShapeM \nitemt{i}{j\!\!+\!1}{\codeblue{\ml$\alpha\mr\beta$\md}, s \!\!+\! \delta^M}} \\[0.2in]
      {\nreduce} & \inferrule{ \ShapeM \nitemt{k}{i}{\codeblue{$\ml\alpha\mr\beta$}, s'} \quad \quad \ShapeP \nitemt{i}{j}{\codeblue{$\ml\gamma\mr$}, s}}{ \ShapeMt \nitemt{k}{j}{\codeblue{$\ml\alpha\mr\beta\ml\gamma\mr$} , s'\!\!+\!s\!\!+\!\xi_{ij}^M}} \\[0.2in]
      {\ncombine} & \inferrule{ \ShapeE \nitemt{0}{i}{\codeblue{$\alpha$},s'} \quad\quad \ShapeP \nitemt{i}{j}{\codeblue{$\ml\beta\mr$}, s}}{ \ShapeE \nitemt{0}{j}{\codeblue{$\alpha\ml\beta\mr$}, s'\!\!+\!s\!\!+\!\xi_{ij}^E}} \\[0.2in]
      {\xtom} & \inferrule { \ShapeP \nitemt{i}{j}{\codeblue{$\ml\alpha\mr$},s} \ \ \ \ \ \ }{ \ShapeM \nitemt{i}{j}{\codeblue{$\ml\alpha\mr$},s\!\!+\!\xi_{ij}^M}} %\\[0.2in]
      & \inferrule { \ShapeMt \nitemt{i}{j}{\codeblue{$\ml\alpha\mr\beta\ml\gamma\mr$},s}}{ \ShapeM \nitemt{i}{j}{\codeblue{$\ml\alpha\mr\beta\ml\gamma\mr$},s}} \\[0.2in]
      {\multileft} & \inferrule{\ShapeMt \nitemt{i}{j}{\codeblue{$\ml\alpha\mr\beta\ml\gamma\mr$}, s} \ \ \ \ \ \ \ \ \ \ \ \ \ }{\ShapeMulti \nitemt{k}{\jnext(k,j)}{ \bml\codeblue{$\md\md\md\ml\alpha\mr\beta\ml\gamma\mr\md\md$}, s\!\!+\!u\cdot \delta^M } } &  $u = (\jnext(k,j)\!-\!j)\!\!+\!(i\!-\!k\!-\!1) \leq 30$\\[0.2in]
      {\multijump} & \inferrule{\ShapeMulti \nitemt{i}{j}{\bml\codeblue{$\md\md\md\ml\alpha\mr\beta\ml\gamma\mr\md\md\md$}, s} \ \ \ \ }{\ShapeMulti \nitemt{i}{\jnext(i,j)}{\bml\codeblue{$\md\md\md\ml\alpha\mr\beta\ml\gamma\mr\md\md\md\md\md$}, s\!\!+\!u\cdot \delta^M }  } & $u = \jnext(i,j) \!\!-\! j$ \\[0.2in]
      {\npair} & \inferrule{ \ShapeH \nitemt{i}{j}{\bml\codeblue{$\md\md\md$},s} \ \ \ \ \ \ }{ \ShapeP \nitemt{i}{j\!\!+\!1}{\codeblue{\ml\md\md\md\mr}, s\!\!+\!\xi_{ij}^H} }\\[0.2in]
      {\single} & \inferrule{ \ShapeP \nitemt{i}{j}{\codeblue{$\ml\alpha\mr$},s} \ \ \ \ \ \ \ \ \ \ \ \ \ \ \ \ \ \ \ \ \ \ \ \ \ \ \ \ \ \ \ \ \ \ \ \ \ \ }{ \ShapeP \nitemt{k}{l}{ \codeblue{\ml\md\md\md$\ml\alpha\mr$\md\md\md\mr},s\!\!+\!\xi_{ij}^S\!\!+\!\omega_{ijlk}^S }} &  $~(x_k, x_{l\!-\!1})~$ match, $(l\!-\!j\!-\!1)\!\!+\!(i\!-\!k\!-\!1) \leq 30$ \\[0.2in]
      {\multi} & \inferrule{ \ShapeMulti \nitemt{i}{j}{\bml\codeblue{$\md\md\md\ml\alpha\mr\beta\ml\gamma\mr\md\md\md$},s}\ \ \ \ \ \ }{ \ShapeP \nitemt{i}{j\!\!+\!1}{\codeblue{$\ml\md\md\md\ml\alpha\mr\beta\ml\gamma\mr\md\md\md\mr$}, s\!\!+\!\xi_{ji}^M\! }} & % $~(x_k, x_{l\!-\!1})~$ match, $u = (l\!-\!j\!-\!1)\!\!+\!(i\!-\!k\!-\!1) \leq 30$
      \\[0.2in]
    \end{tabular}
  }
  \caption {The actual deductive system implemented in \linearfold program.
  Here $\delta^E$, $\delta^M$, $\xi^M_{ij}$, $\xi^E_{ij}$, $\xi^H_{ij}$, $\xi^S_{ij}$, and $\omega^S_{ijlk}$
  are the various energy or scoring parameters ($E$ stands for external loop, $M$ for multiloop, $S$ for single loop, and $H$ for hairpin loop).
  Our \linearfold algorithm can linearize any dynamic programming-based pseudoknot-free RNA secondary structure prediction algorithm.
  The $\jnext(i, j)$ returns the next position after $x_j$ that can pair with $x_i$;
  this is the ``jumping'' trick used in \contrafold and \viennarna to speedup from the theoretical runtime of $O(n^3)$ to
  the empirical  $O(n^{2.6})$
  (see Fig.~\ref{fig:time}). Our final two rules also use this jumping trick in the righthand side loop.
  The only cubic-time rule is \nreduce (intermediate step in multiloop), again inspired by \contrafold source code.
  %% scores the unpaired
  %% nucleotide, and $\xi_{ij}$ scores a base pair.
    \label{fig:realdeduct}}
\end{figure}%% less

% \begin{figure}
%   \centering
%   \hspace{-5em}
%   \scalebox{1.2}{
%   \begin{tabular}{llll}
%     \toprule
%     Method & Global or Local & Time Complexity & \\
%     \midrule
%     CONTRAfold\cite{do+:2006}  & \multirow{3}{*}{Global} & \multirow{3}{*}{$O(n^3)$} & \\
%     RNAstructure\cite{mathews+turner:2006}  &  &  & \\
%     Vienna RNAfold\cite{lorenz+:2011}  &  &  & \\
%     \midrule
%     Rfold\cite{kiryu+:2007}  & \multirow{3}{*}{Local} &  \multirow{3}{*}{$\Omega(nL^2)$}& \\
%     Vienna RNAplfold\cite{bernhart+:2005}  &  &  & \\
%     LocalFold\cite{Lange+:2012}  &  &  & \\
%     \midrule
%     \linearfold  & Global & $O(nb\log b)$ & \\
%     \bottomrule
%   \end{tabular}
%   }
%   \caption{A summary of time complexity of different local/global RNA secondary
%   structure prediction algorithms. \label{fig:related}}
% \end{figure}
