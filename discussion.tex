% !TEX root = \main.tex
\vspace{-.4cm}
\section{Discussion}
\vspace{-0.1cm}

\subsection{Summary}
% In this paper, we present LinearPartition, which inherits LinearFold main idea and applies it to partition function and base pairing probability calculation, and leads to a small accuracy improvement in both linear runtime and linear memory space. 
% LinearPartition reduces classical cubic runtime by pruning states with lower energy. 
% Although it filters some substructure, but only the ones with worse free energy are given up, and results in a similar partition function as exact search. 
% LinearPartition is 10× faster than Vienna RNAfold for the longest sequence (about 3000 nucleotides) in the dataset. 
% Not only being fast, LinearPartition is as accurate as Vienna RNAfold when comparing MEA and ProbKnot output structure. 
% Surprisingly, even though LinearPar- tition uses an inexact search, it achieves better accuracy on longer families (16S and 23S rRNA).

The classical McCaskill (1990) algorithm for partition function and base pairing probabilities calculations 
are widely used in many studies of RNA sequences, 
but its application has been impossible for long sequences 
(such as full length mRNA)
due to its cubic runtime.
%because of the limitation imposed by their cubic scaling.
To address this issue, we present LinearPartition, a linear-time algorithm that dramatically reduces the runtime without sacrificing output quality. % of partition function and base pairing probabilities calculations.
We confirm that:
\begin{enumerate}
  \vspace{-0.2cm}
  \setlength{\itemsep}{0pt}
 \item 
 \linearpartition takes only linear runtime and memory usage, and is orders of magnitude faster on longer sequences (Fig.~\ref{fig:runtime}); %, for example, 
% about 11$\times$ faster than \viennarnafold on {\it H.~pylori} 23S rRNA (2,968~{\it nt}). 
%For example, it enjoys a 2,771$\times$ speedup (2.5 days vs.~1.3~min.)
%over \contrafold on the longest sequence (32,753~{\it nt}) that \contrafold can run in the RNAcentral dataset (Fig.~\ref{fig:runtime}).
 \item
 The base pairing probabilities produced by \linearpartition are better correlated with the ground truth structures on average (Figs.~\ref{fig:ensemble}--\ref{fig:example}); 
% For instance, 
% \linearpartition's ensemble distribution leads to 211.4 (7.3\%) more correctly predicted nucleotides %decrease in ensemble defect
% on \ecoli 23S rRNA compared with \viennarnafold, 
% and 47.1 (9.3\%) more on {\it C.~ellipsoidea} Group I Intron 
% (Figs.~\ref{fig:ensemble} and~\ref{fig:example}).
 \item
When used with downstream structure prediction methods such as MEA and ThreshKnot,
  LinearPartition's base pair probabilities have similar overall accuracy (or even a small improvement on MEA structures) compared with \rnafold,
  as well as better accuracies on longer families and long-distance base pairs (Fig.~\ref{mea});
%   and slightly better accuracies on longer families. 
% \linearpartition is also substantially better on long-distance base pairs (500+~\nts apart) in both MEA and ThreshKnot predictions (Fig.~\ref{mea}).
\item
 \linearpartition has a reasonable approximation quality (Figs.~\ref{fig:partition}--\ref{fig:rmsd}) in terms of \RMSD. 
 
%and can serve multiple downstream tasks.
% Although filtering out some structures, the ensemble free energy change of \linearpartition is either the same or only slightly smaller than \viennarnafold
% e.g., the largest fraction of total free energy change is 2.5\% in the ArchiveII dataset 
% (Fig.~\ref{fig:partition}).  
% Additionally, RMSD of base pairing probabilities between \linearpartition and \viennarnafold is small 
% e.g., the largest RMSD in the ArchiveII dataset is 0.065 
% (Fig.~\ref{fig:rmsd}).
% \item
 % (5) With increasing beam size, \linearpartition gradually approaches the classical McCaskill algorithm;
 %  both the difference in ensemble free energy change and RMSD quickly shrink to 0 (Fig.~\ref{fig:beamsize}).
%  and eventually produces identical outputs.
%%   averaged RMSD decreases. The change is more pronunced from beam size 20 to 100. 
%% Above 100, averaged RMSD is smaller than 0.05, and overall PPV and Sensitivity are stable. 
%% For tmRNA, PPV and Sensitivity increase with beam size and are very close to \viennarnafold at beam size 200. 
%% % Other families, especially for long families such as 16S rRNA, accuracy may drop when increase beam size.
%% But for 16S rRNA, accuracy drops with an increase beam size (Fig.~\ref{beamsize}).
 \end{enumerate}

  % \subsection{Analysis}
%  \smallskip
\vspace{-0.2cm}
There are two possible reasons why our approximation 
results in better base pairing probabilities: 
%in the sense of correlating with ground truth structure:
\begin{enumerate}
  \vspace{-0.2cm}
  \setlength{\itemsep}{0pt}%
 
\item
  This is consistent with the findings in \linearfold, where
  approximate folding via beam search yields more accurate structures.


%% It has been studied that 
%% a structure containing 86.1\% of the ground truth base pairs can be found
%% in a free energy gap of 4.8\% from the optimal structure~\cite{zuker+:1991,mathews+:1999}.
%% With well-designed heuristic,
%% \linearpartition captures the bulk of the free energy of the ensemble, 
%% especially sums up the suboptimal structures within a small gap from the MFE structure.

\item
%% Classical partition function sums up all possible secondary structures,
%% but in reality not all these structures exists at equilibrium.
\linearpartition's pruning of low-probability (sub)structures
has a ``regularization'' effect.
It eliminates some noise in the current energy model 
which is highly inaccurate, especially for long-distance interactions.
\end{enumerate}

\vspace{-0.2cm}
The success of \linearpartition is arguably more striking than \linearfold,
since the former needs to sum up exponentially many structures
that capture the bulk part of the ensemble free energy,
while the latter only needs to find one single optimal structure.

\vspace{-0.2cm}
\subsection{Extensions}

Our work has potential extensions.
\begin{enumerate}
  \vspace{-0.2cm}
  \setlength{\itemsep}{0pt}%
\item 
% Many ncRNAs function by interacting with other RNA sequences by base pairing.
  Existing methods and tools for % these problems
  bimolecular and multistrand 
  base pairing probabilities as well as accessibility computation % could be accelerated and improved.
%calculating two-strand (bimolecular) or multi-strand folding partition functions and
%base pairing probability matrices
\cite{chitsaz+:2009, bernhart+:2006b, Dirks+:2007, DiChiacchio+:2016} are rather slow, %suffer from slowness, 
which limits their applications on long sequences. 
\linearpartition will likely provide a much faster solution for these problems. 
% and will have immediate impact on the ability to predict bimolecular or multi-strand structures 
% and also providing additional structure information (both intra- and inter-molecular pairs) to users.

\item 
We will linearize the partition function-based heuristic methods for pseudoknot prediction such as 
% ProbKnot, 
IPknot and Dotknot. 
These heuristic methods use rather simple criteria to choose pairs from the base pairing probability matrix,
and their runtime bottleneck is $O(n^3)$-time calculation of the base pairing probabilities.
% For example, 
% IPknot %first computes base pairing probabilities 
% selects base pairs by solving an Integer Linear Program (ILP)
% to maximize the total probabilities of chosen pairs
% with well-designed constrains.
% Actually the first step of computing base pairing probabilities
% takes the vast majority of total computation time; e.g., on 23S rRNAs it accounts for 99.3\% of total time. % takes much more time.
% This might appear as $O(n^2)$ in the worst case, 
% but since the linear-time beam search used in \linearpartition only returns $O(nb)$ pairs where $b$ is the constant beam size, 
% this second step is still $O(n)$, 
% giving an overall linear-time method, LinearProbKnot. 
With \linearpartition we can overcome the costly bottleneck %$O(n^3)$-time computation
and get an overall much faster tool. %, such as IPknot or FastDotKnot.
% We can similary get FastDotKnot, etc.
% With the promising results of \linearpartition, 
% we believe \linearpartition-powered IPknot (or DotKnot) might be as accurate as, 
% if not more accurate than, 
% their original $O(n^3)$ versions.

\item 
We can also speed up stochastic sampling of RNA secondary structures from Boltzmann distribution.
The standard stochastic sampling algorithm runs in worst-case $O(n^2)$ time~\cite{Ding+Lawrence:2003},
but relies on the classical $O(n^3)$ partition function calculation.
% The slowness of 
% which becomes the bottleneck for sampling on long sequences.
With \linearpartition, 
we can apply stochastic sampling to full length sequences such as mRNAs, 
and compute their accessbility based on sampled structures.

\end{enumerate}


