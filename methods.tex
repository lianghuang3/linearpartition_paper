% !TEX root = main.tex

% \setcounter{figure}{0}
% \renewcommand{\thefigure}{Online Method \arabic{figure}}
% \setcounter{table}{0}
% \renewcommand{\thetable}{Online Method \arabic{table}}

\section*{Methods}

\small

\subsection*{Datasets}

We use sequences from two datasets, ArchiveII and RNAcentral.
The archiveII dataset 
(available in \url{http://rna.urmc.rochester.edu/pub/archiveII.tar.gz}) is % \cite{sloma+mathews:2016}, 
a diverse set with 3,857 RNA sequences and their secondary structures.
It is first curated in the 1990s to contain sequences with structures that were well-determined by comparative sequence analysis~\cite{mathews+:1999}% [72] 
and updated later with additional structures~\cite{sloma+mathews:2016}. % [96].  
% and is available in \url{http://rna.urmc.rochester.edu/pub/archiveII.tar.gz}.
We remove 957 sequences that appear both in the ArchiveII and the S-Processed datasets~\cite{Andronescu+:2007}, because CONTRAfold uses S-Processed for training. 
We also remove all 11 Group II Intron sequences 
because there are so few instances of these that are available electronically.
Additionally, we removed 30 sequences in the tmRNA family because the annotated structure for each of these sequences contains fewer than 4 pseudoknots, 
which suggests the structures are incomplete. 
These preprocessing steps lead to a subset of ArchiveII with 2,859 reliable secondary structure 
% RNA sequence and structure pairs 
examples 
distributed in 9 families. 
See~\ref{tab:archiveII} for the statistics of the sequences we use in the ArchiveII dataset.
% But since CONTRAfold (v2.02) machine-learned model is trained on the S-Processed dataset \cite{},
% we removed those overlap sequences. 
% As in \linearfold paper, we also remove those sequences CONTRAfold (v2.02) used for training.
% The remaining dataset contains 2889 RNA sequences from 9 families, 
% with average length 222 $nt$ and max length 2968 $nt$.
Moreover, we randomly sampled 22 longer RNA sequences (without known structures) from RNAcentral 
% (The RNAcentral Consortium, 2017) 
(\url{https://rnacentral.org/}),
with sequence lengths ranging from 3,048~{\it nt} to 244,296~{\it nt}.
For the sampling, we evenly split the range from $3,000$ to $244,296$ (the longest) into 24 bins by log-scale, and for each bin we
randomly select a sequence (there are bins with
no sequences).
% (Homo Sapiens Transcript NONHSAT168677.1, from the NONCODE database (Zhao et al., 2016)).
% We run all experiments 
% % (compiled by GCC 4.9.0) 
% on a Linux machine, 
% with 2.90GHz Intel Core i9-7920X CPU and 64G memory.

To show the approximation quality on random RNA sequences, 
we generated 30 sequences with uniform distribution over \{A, C, G, U\}.
% The lengths of these sequences are 100, 200, ..., 3000.
The lengths of these sequences are spaced in 100 nucleotide intervals from 100 to 3,000.
% with lengths range from 100~{\it nt} to 3,000~{\it nt}. 


\subsection*{Baseline Software}
% We present two versions of \linearpartition, \linearpartitionv and \linearpartitionc.
% \linearpartitionv uses the experiment-based thermodynamic parameters~\cite{mathews+:1999,Mathews+:2004,xia+:1998}
% as implemented in 
We use two baseline software packages: 
(1) \viennarnafold %~\cite{lorenz+:2011}
(Version 2.4.11) 
from 
\url{https://www.tbi.univie.ac.at/RNA/download/sourcecode/2_ 4_x/ViennaRNA-2.4.11.tar.gz} 
and 
(2) \contrafold %~\cite{do+:2006}
(Version 2.0.2) 
from
\url{http://contra.stanford.edu/}.
\viennarnafold is a widely-used RNA structure prediction package,
while \contrafold is a successful machine learning-based RNA structure prediction system.
Both provide partition function and base pairing probability calculations based on 
the classical cubic runtime algorithm.
Our comparisons mainly focus on the systems with the same model, 
i.e., \linearpartitionv vs. \viennarnafold and \linearpartitionc vs. \contrafold.
In this way the differences are based on algorithms themselves rather than models.
% bugs in contrafold
We found a bug in \contrafold by comparing our results to CONTRAfold, 
which led to overcounting multiloops in the partition function calculation.
We corrected the bug, and all experiments are based on this bug-fixed version of \contrafold.


\subsection*{Evaluation Metrics and Significance Test}

Due to the uncertainty of base-pair matches existing in comparative analysis
and the fact that there is fluctuation in base pairing at equilibrium,
we consider a base pair to be correctly predicted if it is also displaced by one
nucleotide on a strand~\cite{mathews+:1999}.
Generally, if a pair $(i,j)$ is in the predicted structure, we consider it a
correct prediction if one of $(i,j)$, $(i-1,j)$, $(i+1,j)$, $(i,j-1)$, $(i,j+1)$ is in the
ground truth structure.
% We also report the accuracy using exact base pair matching instead of this
% method, in Figure~\ref{tab:accuracy_nos}. 
%To evaluate the accuracy, 
% Both sensitivity and PPV are reported.
%Positive Predictive Value
%(PPV), 

We use Positive Predictive Value (PPV)
and sensitivity 
as accuracy measurements. 
Formally, denote $\vecy$ as the predicted structure and $\vecy^{*}$ as the ground
truth, we have:
% $\ppv = \frac{\text{true positives}}{\text{true positives} +
%   \text{false positives}} = \frac{|\vecy \cap \vecy^*|}{|\vecy|} $
% $ \sens = \frac{\text{true positives}}{\text{true positives} +
%   \text{false negatives}} =
% \frac{|\vecy \cap \vecy^*|}{|\vecy^*|}$

$$\ppv = \frac{\#_{\text {TP}}}{\#_{\text {TP}} + \#_{\text {FP}}}  = 
\frac{|\pairs(\vecy) \cap \pairs(\vecy^*)|}{|\pairs(\vecy)|} $$

$$ \sens = \frac{\#_{\text {TP}}}{\#_{\text {TP}} + \#_{\text {FN}}}  =
\frac{|\pairs(\vecy) \cap \pairs(\vecy^*)|}{|\pairs(\vecy^*)|}$$
where $\#_{\text {TP}}$ is the number of true positives (correctly predicted pairs),
$\#_{\text {FP}}$ is the number of false positives (wrong predicted pairs)
and $\#_{\text {FN}}$ is the number of false negatives (missing ground truth pairs).

We test statistical significance using a paired, two-sided permutation test~\cite{Aghaeepour+Hoos:2013}.
We follow the common practice, choosing $10,000$ as the repetition number
and $\alpha=0.05$ as the significance threshold.
% following previous work~\cite{Aghaeepour+Hoos:2013}.

\subsection*{Curve Fitting}
We determine the best exponent $a$ for the scaling curve $O(n^a)$ for each data series in Figures~\ref{fig:linearpairs} and \ref{fig:runtime}.
Specifically, we use $f(x) = a x + b$ to fit the log-log plot of those series in Gnuplot;
e.g., fitting $\log t_n = a \log n + b$, where $t_n$ is the running time on a sequence of length $n$,
so that $t_n = e^b n^a$.
Gnuplot uses the nonlinear least-squares Marquardt-Levenberg algorithm.


\section*{Code availability}\label{code}

Our \linearpartition source code can be downloaded  from\\ 
{\tt\url{https://github.com/LinearFold/LinearPartition}}.

\section*{Data availability}

%\small

The data that support the findings of this study are available from the corresponding author upon request.

%% The data sets used in this study are  available online:
%% ArchiveII~data~set: {\small\url{http://rna.urmc.rochester.edu/pub/archiveII.tar.gz}};
%% RNAcentral data~set: \url{https://rnacentral.org}.
%by requesting from the corresponding
%author.
